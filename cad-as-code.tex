Some programmers have a bit of a stereotype for being extremely caring and coddling about their code. This, coupled with the inherent amount of flexibilty in method afforded by writing code, means that there seems to have developed an entire field surrounding programming styles and best practices that rivals and arguably supersedes rhetoric in its nuance for effective writing.

The nature of programming itself also requires a good deal of thinking about many abstract concepts in a somewhat concrete form: heirarchies, modules, maintenance, documentation; and many abstract needs: stability, extensability, readability. Nowhere have seen such dedication to these excellent engineering pursuits laid out in such a way that is actually easy to comprehend and apply in daily life (as opposed to miles of standard operating procedures which an engineer has scant hope of doing, let alone reading in entirety). 

I spent many of my formative years as a programmer, so


\section{Naming}
Naming isn't as big of a deal in CAD as it is with code, where the majority of elements must be created by typing a name for an object, whereas in CAD entity creation is accomplished through graphical means. This, coupled with the inherent modularity afforded by creating parts and assemblies, 

I fear, though, that the increased prevalence of OnShape and its part studios will pose a similar need for rigorous naming, as the inherent modularity of parts is stripped away in favor of larger 'studios'.

\section{Relationships}
% Alt title: how to avoid having oedipus complexes

When you show a CAD user the power afforded by in-context modeling, or top-down modeling, the power can be overwhelming. The ability to create parts that reference each other, driving each little bit
